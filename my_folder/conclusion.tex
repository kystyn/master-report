\chapter*{Заключение} \label{ch-conclusion}
\addcontentsline{toc}{chapter}{Заключение}	% в оглавление 

В результате проделанной работы был разработан программный продукт, который позволяет распознавать структуру химической молекулы с применением object detection подхода с точностью 81.8\%. Было проведено исследование, результатом которого стало понимание особенностей работы object detection модели Detectron2 Faster R-CNN на изображениях химических молекул.

Учитывая хорошую разделимость этих данных, можно сделать вывод, что модуль классификации Detectron2 достаточно плохо справляется с возложенной на него обязанностью. Однако регрессор баундбоксов работает очень качественно, в связи с чем можно дать общую рекомендацию по применению таких сетей.

Данная сеть обучается очень быстро и не склонна существенно улучшать качество работы после первых нескольких тысяч итераций, несмотря на то, что loss-функция выходит на асимптоту только приблизительно к статысячной итерации. Стоит обучать такую сеть на поиск баундбоксов, а задачу классификации отдавать в следующую сеть, скорее всего простую по своей структуре. Таким образом, удастся качественно выполнить поставленную задачу и, несмотря на кажущуюся громоздкость и дублирование функциональности двух моделей, практически не потерять в производительности инференса.

Также благодаря такому подходу можно добавить распознавание новых особенностей связей и меток атомов путём минимальных усилий: добавления категорий в классификаторы и сборщик молекул. Обучение классификаторов с нуля даже на CPU занимает не более, чем несколько десятков минут.

Кроме того, в построенном решении найдены проблемные места и даны рекомендации по их устранению и дальнейшим исследованиям: сеть иногда распознаёт очень большое количество баундбоксов на метке атома кислорода и в целом склонна к некорректному распознаванию меток атомов. Это связано с тем, что образцов меток атомов у сети во много раз меньше, чем образцов связей, в связи с чем сеть недоучивается распознавать метки атомов, и вместо них видит связи. Для исправления этой ситуации необходимо реализовать генератор химических соединений, который позволит создавать химически корректные молекулы с большим числом атомов с указываемыми метками -- преимущественно азот, кислород и водород.
